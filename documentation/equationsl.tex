\RequirePackage{shellesc}
\documentclass{scrartcl}
\KOMAoptions
  {
    fontsize=26pt,
    paper=a1,
    pagesize=pdftex,
    DIV=calc,
    headings=standardclasses,
    headings=small,
    twoside=on,
    BCOR=1cm 
  }
  
\usepackage{adrianschneider}
\DeclareMathOperator{\sign}{sign}
\DeclareMathOperator{\Langevin}{L}
\newcommand{\He}{H_\text{e}}
\newcommand{\Man}{M_\text{an}}
\newcommand{\Msat}{M_\text{sat}}
\newcommand{\Mirr}{M_\text{irr}}
\newcommand{\Mrev}{M_\text{rev}}
\newcommand{\textref}[1]{\text{(\ref{#1})}}
\newcommand{\eqr}[1]{\underset{\textref{#1}}{=}}
\definecolor{gray}{RGB}{92, 92, 92}
  
\begin{document} 
\color{gray}
\begin{align}
	\He &= H + \alpha M \quad &&\text{Effektives Feld} \label{eq:h_eff}\\
	\Man &= \Msat \Langevin\left(\frac{\He}{a}\right) \quad &&\text{Anhysteretische Magnetisierung}  \label{eq:m_an}\\
	\frac{\dif\Mirr}{\dif \He} &= \frac{\Man - \Mirr}{k \sign\left(\frac{\dif H}{\dif t}\right)} \quad &&\text{Pinning} \label{eq:pinning}\\
	M &= \Mrev + \Mirr \quad &&\text{Gesamte Magnetisierung} \label{eq:m_gesamt}\\
	\Mrev &= c(\Man - \Mirr) \quad &&\text{Irreversible Magnetisierung} \label{eq:m_rev}
\end{align}


\begin{align}
	\frac{\dif M}{\dif \He} &= (1-c) \frac{\dif \Mirr}{\dif \He} + c \frac{\dif \Man}{\dif \He} \label{eq:dm_dhe}
\end{align}

\begin{equation*}
	\begin{aligned}
		\frac{\dif M}{\dif H} &= \frac{\dif M/\dif \He}{1 - \alpha \dif M/\dif \He}
	\end{aligned}
\end{equation*}

\begin{equation*}
\begin{aligned}
	\frac{\dif B}{\dif H} = \mu_0 \frac{1 + (1-\alpha)\dif M/\dif \He}{1 - \alpha \dif M/\dif \He}. \label{eq:dB_dH}
\end{aligned}
\end{equation*}

\begin{equation*}
	\frac{\dif H}{\dif B} = \frac{1}{\mu_0} \frac{1 - \alpha \dif M/\dif \He}{1 + (1-\alpha)\dif M/\dif \He}
\end{equation*}

\begin{labeling}{$\Msat$}
	\item[$\alpha$]{Interdomänenkopplung}
	\item[$a$]{Domänenwanddichte}
	\item[$\Msat$]{Sättigungsmagnetisierung}
	\item[$k$]{Pinning-Energie}
	\item[$c$]{Magnetisierungsreversibilität}
\end{labeling}

\begin{equation}
\begin{aligned}
	\frac{\dif y}{\dif t} &= f(y, t)
\end{aligned}
\end{equation}

\begin{equation*}
\begin{aligned}
	\text{Nächster Zeitpunkt} && t_n &= t_{n-1} + h \\
	\text{i-ter Hilfswert} && k_i &= f\left(y_{n-1} + h \sum_{l=1}^m \beta_{i,l}k_l, t_{n-1} + \alpha_i h\right) \\
	\text{Nächster Wert} && y_n &= y_{n-1} + h\sum_{l=1}^m y_l k_l
\end{aligned}
\end{equation*}

\begin{equation*}
	\begin{array}{c|ccc}
	\alpha_1 	& \beta_{1,1}   & \cdots & \beta_{1,m} \\
	\alpha_2 	& \beta_{2,1}   & \cdots & \beta_{2,m} \\
	\vdots      & \vdots        & \ddots & \vdots        \\
	\alpha_m    & \beta_{m,1}   & \cdots & \beta_{m,m} \\
	\hline
	            & \gamma_1      & \cdots & \gamma_m
\end{array}.
\end{equation*}


\begin{equation}
\renewcommand\arraystretch{1}
\arraycolsep10
\begin{array}{c|cccc}
	0 &&&& \\
	\tfrac{1}{2} & \tfrac{1}{2} &&& \\
	\tfrac{1}{2} & 0 & \tfrac{1}{2} && \\
	1 & 0 & 0 & 1 &\\
	\hline
	& \tfrac{1}{6} & \tfrac{1}{3} & \tfrac{1}{3} & \tfrac{1}{6}
\end{array}
\end{equation}


\end{document}
