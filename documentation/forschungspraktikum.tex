\documentclass{scrartcl}
\KOMAoptions
  {
    fontsize=12pt,
    paper=a4,
    pagesize=pdftex,
    DIV=calc,
    headings=standardclasses,
    headings=small,
    twoside=on,
    BCOR=1cm
  }
  
\usepackage{adrianschneider}
\DeclareMathOperator{\sign}{sign}
\DeclareMathOperator{\Langevin}{L}
\newcommand{\He}{H_\text{e}}
\newcommand{\Man}{M_\text{an}}
\newcommand{\Msat}{M_\text{sat}}
\newcommand{\Mirr}{M_\text{irr}}
\newcommand{\Mrev}{M_\text{rev}}
\newcommand{\textref}[1]{\text{(\ref{#1})}}
\newcommand{\eqr}[1]{\underset{\textref{#1}}{=}}
  
\begin{document}
\section{Jiles-Atherton-Modell}
Das Jiles-Atherton-Modell ist ein physikalisch motiviertes Modell zur Beschreibung der Hysterese von magnetischen Materialien.
\subsection{Konstituierende Gleichungen}
\begin{align}
	\He &= H + \alpha M \quad &&\text{Effektives Feld} \label{eq:h_eff}\\
	\Man &= \Msat \Langevin\left(\frac{\He}{a}\right) \quad &&\text{Anhysterische Magnetisierung}  \label{eq:m_an}\\
	\frac{\dif\Mirr}{\dif \He} &= \frac{\Man - \Mirr}{k \sign\left(\frac{\dif H}{\dif t}\right)} \quad &&\text{Pinning} \label{eq:pinning}\\
	M &= \Mrev + \Mirr \quad &&\text{Gesamte Magnetisierung} \label{eq:m_gesamt}\\
	\Mrev &= c(\Man - \Mirr) \quad &&\text{Irreversible Magnetisierung} \label{eq:m_rev}
\end{align}
In diesen fünf Modellgleichungen kommen fünf Modellparameter vor:
\begin{labeling}{$\Msat$}
	\item[$\alpha$]{Interdomänenkopplung}
	\item[$a$]{Domänenwanddichte}
	\item[$\Msat$]{Sättigungsmagnetisierung}
	\item[$k$]{Pinning-Energie}
	\item[$c$]{Magnetisierungsreversibilität}
\end{labeling}
\subsection{Herleitung der Differentiale}
Ziel ist es nun, einen differentiellen Zusammenhang zwischen $M$ und $H$ zu finden. Dazu:
\begin{equation}
	M \underset{\textref{eq:m_gesamt}}{=} \Mrev + \Mirr \underset{\textref{eq:m_rev}}{=} c (\Man - \Mirr) + \Mirr = c\Man + (1-c)\Mirr.
\end{equation}
Es gilt also
\begin{equation}
	\dif M = (1-c) \dif\Mirr + c\dif \Man \label{eq:dm},
\end{equation}
weiterhin stimmt jeweils
\begin{align}
	\dif \Mirr &= \frac{\dif \Mirr}{\dif \He}\dif \He \\
	\dif \Man &= \frac{\dif \Man}{\dif \He}\dif \He\\
	\dif \He & \underset{\textref{eq:h_eff}}{=} \dif H + \alpha \dif M.
\end{align}
Man erhält nun für $\dif M$ zusammengefasst:
\begin{align}
	\begin{split}
		\dif M &\eqr{eq:dm} (1-c) \dif\Mirr + c\dif \Man \\
				&= (1-c) \frac{\dif \Mirr}{\dif \He}\dif \He + c \frac{\dif \Man}{\dif \He}\dif \He,
	\end{split}
\end{align}
in dieser Darstellung sind nun die aus den konstituierenden Gleichungen \ref{eq:m_an} und \ref{eq:pinning} leicht zugänglichen Größen $\dif \Mirr/\dif \He$ und $\dif \Man/\dif \He$ enthalten.
\begin{align}
	\Rightarrow \frac{\dif M}{\dif \He} &= (1-c) \frac{\dif \Mirr}{\dif \He} + c \frac{\dif \Man}{\dif \He} \label{eq:dm_dhe}
\end{align}
Abhängig von der Formulierung des elektrodynamischen Problems können nun zwei verschiedene Formulierungen gewählt werden.
\paragraph{Formulierung mit H-Feld}
Soll das Problem abhängig vom H-Feld gewählt werden, formuliert man $\dif \He$ wie folgt
\begin{equation}
	\dif \He \eqr{eq:h_eff} \dif H + \alpha \dif M,
\end{equation}
also
\begin{align}
	\dif M &= \frac{\dif M}{\dif \He} \dif \He \\
	&= \frac{\dif M}{\dif \He} \left(\dif H + \alpha \dif M\right) \\
	\Rightarrow \frac{\dif M}{\dif H} &= \frac{\dif M / \dif \He}{1 - \alpha \dif M /\dif \He}
\end{align}
\paragraph{Formulierung mit B-Feld} Bei der Formulierung mit der magnetischen Flussdichte formuliert man mit $B = \mu_0(H + M)$
\begin{align}
	\dif \He &= \frac{\dif B}{\mu_0} - \dif M + \alpha \dif M \\
		&= \frac{\dif B}{\mu_0} - (1 - \alpha) \dif M
\end{align}
Für $\dif M/\dif B$ erhält man nun
\begin{align}
	\dif M &= \frac{\dif M}{\dif \He} \dif \He \\
	&=\frac{\dif M}{\dif \He}\left(\frac{\dif B}{\mu_0} - (1 - \alpha) \dif M\right) \\
	\Rightarrow \frac{\dif M}{\dif B} &= \frac{1}{\mu_0}\frac{\dif M/\dif \He}{1 + (1-\alpha) \dif M/\dif \He}
\end{align}
\end{document}